\documentclass[11pt]{article}
\usepackage{geometry} % see geometry.pdf on how to lay out the page. There's lots.
\usepackage{hyperref}
\usepackage{graphicx}
\usepackage{gensymb}
\usepackage[affil-it]{authblk}
\usepackage[toc,page]{appendix}
\usepackage{pifont}
\usepackage{amsmath}
\usepackage{amsthm}

\usepackage{float}

\newtheorem{theorem}{Theorem}
\newtheorem{corollary}{Corollary}
\newtheorem{conjecture}{Conjecture}
\newtheorem{proofsketch}{Proof Sketech}

\newcommand\numberthis{\addtocounter{equation}{1}\tag{\theequation}}


\usepackage{draftwatermark}


\SetWatermarkText{DRAFT}
\SetWatermarkScale{6}
\SetWatermarkLightness{0.95}

% \geometry{letter} % or letter or a5paper or ... etc
% \geometry{landscape} % rotated page geometry

% See the ``Article customise'' template for come common customisations

\title{Untwisting the Tetrahelix (v0.5)}
\author{Robert L. Read \texttt{read.robert@gmail.com} \and
  Robert Gatliff \texttt{robert@toubat.org}
}


\date{\today}

%%% BEGIN DOCUMENT
\begin{document}

\maketitle

%% \tableofcontents

\begin{abstract}
  The Boerdijk--Coxeter helix (BC helix, or tetrahelix) is a face-to-face stack of regular tetrahedra.
  Considering the edges of these tetrahedra, the resulting structure is attractive and inherently rigid,
  and therefore interesting to architects, mechanical engineers,
  and robotocists.
  A formula that matches the visually apparent helices forming the outer rails of the tetrahelix is derived which
  is convenient for designers.
  This formula defines the vertices of tetrahelices of varying radius, pitch, and curvature, with the BC helix
  as a special case. 
  An addtional special case is of that of $0$ curvature or rail angle, which generates a \emph{tetrabeam}.
  A particular choice of paramaters defines a novel object, the  minimax member length-difference tetrabeam,
  the \emph{equitetrabeam}.
  Linear interpolation of these parameters with the BC-helix parameters defines a continuum of close-to regular
  tetrahelices of designable curvature, pitch based on a single parameter.
  Utility and use for static and variable geometry truss/space frame design and robotics based on the are discussed.
\end{abstract}


\section{Introduction}

The Boerdijk--Coxeter helix\cite{coxeter1985simplicial} (BC helix), is
a face-to-face stack of tetrahedra that winds about a straight axis.
Because architects, structural engineers, and robotocists are inspired
by and follow such mathematical models but can build structures and
machines of differing or even dynamically changing length, it is
useful to develop the mathematics of structure formed from tetrahedra
where we relax regularity.  The vertices of the tetrahedra lie upon
three helices about the central axis.  The
Tetrobot/Glussbot\cite{TetrobotBook} project uses the regularity of
this geometry to make a tentacle-like robot that can crawl like a slug
or mollusc.  The Tetrobot concept is to use mechanical members, called
actuators, which can change their length, connected by special joints,
called the Song-Kwon-Kim\cite{song2003spherical} or turret joint,
which allow many members to come to a single point.  Such machines can
follow purely regual mathematical models such as the Boerdijk–-Coxeter
helix or the Octet Truss\cite{richard1961synergetic}.

Buckminster Fuller called the BC helix a \emph{tetrahelix}\cite{fuller1982synergetics},
a term now commonly used. In this paper we reserve BC helix to mean the purely regular structre and use \emph{tetrahelix} to refer
to any structure isomophic to a the BC helix, whether regular or not.

\begin{figure}[H] %float with two figures
  \label{series}
  \centering
     \includegraphics[width=0.95\textwidth]{figures/BCHelixCloseUp.png}
     \caption{BC Helix Close-up}
\end{figure}

\begin{figure}[H] %float with two figures
  \label{series}
  \centering
     \includegraphics[width=0.95\textwidth]{figures/TetrahelixSeries.png}
     \caption{A Series of Tetrahelices, BC Helix in Foreground.}
\end{figure}


BC helix does not rest on a plane in a simple way. It is convenient to
be able to ``untwist'' it and form a tetrahelix space frame that has a
flat planar surface. By making length changes in a certain way, we can
untwist a tetrahelix to form a \emph{tetrabeam} which has planar faces
and has, for example, an equilateral triangular profile.  In Figure
\ref{series}, the closest helix is the BC helix.

\section{A User's Formulation of the BC Helix}

If you can choose member lengths, you can form a linear combination of
the equitetrabream lengths and the completely regular lengths of the
tetrahelix, thereby choosing the torsion.  If you are designing a
space frame, this is a static design choice, in a robot, it is a
dynamic choice that can be used to twist the robot and/or exert
torsion on the environment.

Ideally we would have a simple formula for defining the nodes based on
any torsion we choose.  Unfortunately, it is not obvious that a linear
combination of lengths produces a simple formula.  It is a goal of
this paper to relate these two approaches to generating a tetrahelix
continuum.

Coxeter constructs the BC helix\cite{coxeter1985simplicial} as a repeated rotation and translation of the tetrahedra, showing the
rotation is:
\[
\theta = \arccos(-2/3) 
\]
and the translation:
\[
h_{bc} = 1/\sqrt{10}
\]

$\theta$ is approximately $131.81$ degrees.
The angle $\theta$ is the rotation of a \emph{each} tetrahedra.
That is, a yellow tetrahedron is rotated slightly more than a $1/3$ of a revolution to match the face of the red tetrahedra.
$3 \theta - 2\pi$ is the apparent rotation of $V_3$ relative to $V_0$.

From Robert Gray's site, repeating formula by H.S.M. Coxeter:
\begin{equation}
  \label{graycoxeter}
V(n) =
\left [
  \begin{tabular}{c}
   $ r_{bc} \cos(n \theta) $\\
   $ r_{bc} \sin(n \theta) $\\
   $ n h_{bc}  $
  \end{tabular}
  \right ],
\text{where:}
  \begin{tabular}{c}
 $ r_{bc} = \frac{3\sqrt{3}}{10} $\\
 $ h_{bc} = 1/\sqrt{10} $ \\
 $ \theta = \arccos(-2/3) $ \\
  \end{tabular}      
\end{equation}
where $n$ represents each integer numbered node in succession.

This formula defines a helix, but it is not any of the helices of the
BC helix, but rather one that winds three times as rapidly through all
nodes. To a designer of tetrahelices, it is more natural to think of
the three helices which are visually apparent, that is, those three
which are closely approximated by the by the outer edges or rails of
the BC helix.

It is convenient to have a formula that gives us the nodes of just
each colored helix.
\[
H_{BCcolored}(n,c) = V(3n +c)
\]
where $c \in \{0,1,2\}$ specifies which of the rails is being computed.

Such a helix can be written:
\begin{equation}
H_{BCcolored}(n,c) =
\left [
  \begin{tabular}{c}
   $ r_{bc}  \cos((3 \theta - 2 \pi)n + c  \theta $\\
   $ r_{bc} \sin((3 \theta - 2 \pi)n + c  \theta $\\
   $ (n + c/3) 3  h_{bc} $
  \end{tabular}
  \right ],
\text{where:}
  \begin{tabular}{c}
 $ r_{bc} = \frac{3\sqrt{3}}{10} $\\
 $ h_{bc} = 1/\sqrt{10} $ \\
 $ \theta = \arccos(-2/3) $ \\
  \end{tabular}      
\end{equation}

In this formula, integral values of $n$ may be taken as a node number for one rail and used to compute its Cartesian
coordinates. Allowing $n$ to take non-integer values defines a continuous
helix in space which is close to the segmented polyline of the outer tetrahedra edges, and coincides with them at integer
values.
The parameter $c \in \{0,1,2\}$ specifies which of the rails is being computed.

The quantity $ (3 \theta - 2 \pi) \approx 35.43 \degree $, and is the angular shift between $V(n,color)$ and
$V(n+1,color)$. This quantity appears so often below that we call it the ``rail angle rho''. For the BC helix, $\rho_{bc} = (3 \theta - 2 \pi)$.

\begin{figure}[H]
  \label{railanglefig}
     \centering
     \includegraphics[width=0.7\textwidth]{figures/RailAngleGeometry.png}
     \caption{Rail Angle Geometry}
 \end{figure}

Since:
\[ \frac{2 \pi}{\rho_{bc}} \approx 10.16
\]
We can see that there are approximately $10.16$ red, blue or yellow tetrahedra on one rail in a single revolution.
The pitch of the Boerdijk--Coxeter helix of edge length $1$ is the length of three tetrahedra times this number:
\begin{align*}
  &= \frac{3 \cdot h_{bc} 2 \pi }{\rho_{bc}} \\
  &= \frac{3  \sqrt{\frac{2}{5}}  \pi}{\rho_{bc}} \\
  &\approx 9.6392 \\
\end{align*}
The pitch is less than the number of tetrahedra because the tetrahedra are not lined up perfectly.
It is a famous and interesting result that the pitch is irrational, a BC helix never has two tetrahedra
at precisely the same orientation around the $z$-axis. However, this is inconvenient to designers, who
might prefer a rational pitch. For example, a slight irregularity that led to a pitch of precisely 10 tetrahedra
in one revolution would allow an architect to design a column having a basis and a capital in the same relation to the tetrahedra
they touch.

A BC helix has the useful property that every member is precisely the same length. If we relax this, so that the tetrahedra it
comprises are not perfectly regular, then we can twist and curve the tetrahelix into a variety of shapes. This is useful to
the mechanical engineer or robotocist because the structure remains an inherently rigid, omni-triangulated space frame, which
may be expected to be at least somewhat mechanically strong.

\section{Optimal Tetrahelices Have Evenly Spaced Vertices}

We use the term \emph{tetrahelix} to mean any structure made of
vertices and edges which is isomorphic to the BC helix, in which the
vertices lie on three helices, no matter what lengths the edges take
on. One could consider various definitions of optimality for a
tetrahelix, but the must useful to an enginner is to minimize the
maximum difference between any two edges. We call a tetrahelix
\emph{member-length minimax optimal} if there is not tetrahelix of the
same radius and pitch with a smaller maximum edge length difference.

Consider a tetrahelix isomporphic to an BC helix of unbounded
extent. If all three rails do not have same pitch, there is an edge of
unbounded length. So we are justified in talking about the pitch of
the tetrahelix, even though we think of it as three
helices. Similarly, if the axes are not parallel, there is an edge of
unbounded length in the structure, so it is not optimal.

Consider any tetrahelix in which the axes of the helices are parallel
to $z$ but not coincident, but in which all three helices have the
same radius. Consider the projection along the $z$ axis, which will be
a figure of dots and connecting segments in the $xy$ plane. The convex
hull for any one helix projection will be a circle (if its pitch is
irrational) or an polygon if rational. So long as its rail angle is
not $0$, it will have at least two points in the plane. One of these
will be longer and will be made shorter by moving that helix closer to
the midpoint of the other two in the $xy$ plane. Even if the rail
angle is $\pi$, there will be two vertices for each rail, each of
which is connected to both vertices of the other two rails. At lesast
one of these lines is a longest line which will be made shorter by
moving that helix in the $xy$ plane closer to the midpoint of the
other two.

So any any optimal tetrahelix with a rail-angle of greater than $0$,
that is, with any curvature, will have conincident axes.

We call a tetrahelix with $0$ curvature a \emph{tetrabeam}, and
consider it as an important special case and an endpoint of a
continuum of tetrahelices of decreasing curvature.


 Now now that we have coincident axes, the same pitch, we can go on to
 the hard proof about where they occur.

 We show that in fact the nodes must be distributed in even thirds
 along the $z$-axis.

 Note that from the point of view of a single edge, we are on a
 slanted cylinder, when $\rho \neq 0$.  This means for its point of
 view a cross section is an ellipse. So we have to be very careful in
 comparing lengths of edges relative to the tetrahedron, because a
 change in position along the edge changes the length of a line, but
 in a complicated way depending on where it is relative to the
 ellipse.

 \begin{theorem}
   \label{eventhirds}
   A tetrahelix of a given radius and height $h$ in which all nodes are evenly spaced at $h/3$ intervals on the $z$ axis is optimal.
     Any one tetrahedron in a tetrahelix has $1$ rail edge, $2$ one-hop edges connected to the rail and $2$ two-hop edges connected to the rail.
  The edge opposite of the rail edge is a one-hop edge.
  \end{theorem}

 \begin{proof}

   In principle in any tetrahelix where the three helices have any
   displacement along the $z$ axis there are 9 distinct edge classes:
   3 rail edges, a one-hop length between eachof 3 rails, and two hop
   length between each of three rails, where the two-hop length is a
   maximum length between rails (which could equal the one-hop
   length.) However, we have already shown the pitches and the rail
   lengths are equal in any optimal tetrahelix.
   
    Consider the tetrahelix in which the vertices are evenly spaced at
    $h/3$ intervals on the $z$ axis. Every edge is either a rail edge,
    or it makes one hop, or it makes two hops. All of the one-hop
    edges are equal length.  All of the two-hop edges are equal
    length.
    
    Any displacemnt along the $z$ axis of any rail increases the
    length of one or two two-hop edges and shortens the length of one
    or two one-hop edges.  This increases the minimax distance no
    matter what the rail edge length is, since all rail edges are the
    same length. Therefore, an evenly spaced tetrahelix is the unique
    optimal for any given radius and height.
  \end{proof}
  

Note that based on \ref{eventhirds}, we are justified in classifying edge lengths as \emph{rail},\emph{one-hop}, or
\emph{two-hops}. The one-hop edges are the edges between closest on the $z$-axis, and the two-hop edges are those that hop over a vertex.

By \ref{eventhirds} every optimal tetrahelix has vertices lying on helices expressible in the form:
\[
V_{optimal}(n,c) =
\left [
  \begin{tabular}{c}
   $ r \cos(n \alpha +  c 2 \pi /3)$\\
   $ r \sin(n \alpha +  c 2 \pi /3) $\\
   $ \frac{d(n +c / 3)}{3}   $
  \end{tabular}
  \right ],
\text{where:}
\begin{tabular}{c}
  $c \in \{0,1,2\}$
  \end{tabular}      
\]
where we have not yet investigated in the general case the relationships beteween $\alpha$, $r$, and $d$ in this formulation.
However, we understand that when $\alpha = 0$, the helices are degenerate, having curvature of $0$, and
we have the equitetrabeam.


This formulation $V(n,c)$ above is valuable, but obscures the essentially fact that the red, yellow, and blue helices distributed
about the central $z$ axis $120\degree$ from each other.
In order to rewrite this expression with an explicit rotation of $2\pi/3$, we expand 
the expression and seek to isolate the term $c2\pi/3$.
\begin{align*}
  \rho_{bc} n + c \theta  &=   \text{\{we aim for 3 in denominator, so we split...\}} \\
    (3 \theta - 2 \pi)n + (c/3)  (\theta /3)  &=   \text{\{we want $2\pi$ in numerator, so add canceling terms...\}} \\
  (3 \theta - 2 \pi)n + (c/ 3) (3 \theta - 2 \pi  + 2 \pi) &=  \text{\{associate...\}} \\
  (3 \theta - 2 \pi)n + (c/ 3) ((3 \theta - 2 \pi)  + 2 \pi) &=  \text{\{distribute...\}} \\  
  (3 \theta - 2 \pi)n + (c / 3) (3 \theta - 2 \pi)  + c 2 \pi /3 &=  \text{\{definition of $\rho_{bc}$...\}} \\
  \rho_{bc} n + (c / 3) \rho_{bc}  + c 2 \pi /3 &=  \text{\{collect like factors...\}} \\  
  \rho_{bc} (n + c/3)  + c 2 \pi /3  \\
\end{align*}
Now the the term on the left is the only one that depends on the scalar $n$. We use this to a create
a new formulation $H_{BCsymmetric}(n,c) = H_{BCcolored}(n,c)$

The expression $n+c/3$ will now occur so often that we call it the ``c($\kappa$)olored number'' and we use the variable $\kappa$ to represent it: $\kappa = n+c/3$.
Recall that $c \in \{0,1,2\}$, but $n$ and $\kappa$ are are continuous (rational or real-valued.)

\begin{equation}
H_{BCsymmmetric}(n,c) =
\left [
  \begin{tabular}{c}
   $ r  \cos(\rho_{bc} \kappa  + c 2 \pi /3) $\\
   $ r  \sin(\rho_{bc} \kappa  + c 2 \pi /3) $\\
   $ \kappa 3  h_{bc} $
  \end{tabular}
  \right ],
\text{where:}
  \begin{tabular}{c}
 $\kappa = n + c/3$ \\
    $\rho_{bc} = (3 \theta - 2 \pi)$ \\
    $ \theta = \arccos(-2/3) $ \\
    $ h_{bc} = 1/\sqrt{10} $ \\    
  \end{tabular}      
\end{equation}

\section{Parametrizing Tetrahelices via Rail Angle}

We seek a formula to generate optimal tetrahelices that accepts a
parameter that allows us to choose the tetrahelix conveniently. The
pitch of the helix is an obious choice, but is not defined when the
curvature is $0$, and important special case. The radius or the axial
distance between two nodes on the same rail are obvious choices, but
perhaps the clearest choice is to build formula that takes as its
input the ``rail angle'' $\rho$. We define $\rho$ to be the angle
formed in the X,Y plane $\angle R_i O R_{i+1}$ projecting out the $z$
axis and sighting along the positive $z$ axis. In other words, $\rho$
controls how far a rail edge of a tetrahelix deviates from being
parallel with the axis, or the ``twistiness'' of tetrahelix. Ideally
we will treat a positive angle as creating a clockwise tetrahelix and
a negative as creating a counter-clockwise helices.

Please refer back to Figure \ref{railanglefig}.

 These quantities are related by the expression:

\begin{align*}
  1^2 &= d^2 + (2 r \sin{ \rho / 2})^2 \\
  1 &= d^2 + 4 r^2 (\sin{ \rho / 2})^2 \\
  d^2 &= 1 - 4 r^2 (\sin{ \rho / 2})^2    \numberthis  \label{railangle} \\
\end{align*}

Checking the important special case of the BC helix, we find that this equation
indeed holds true (treating $d$ in this equation as $3 h_{bc}$ as defined by
Gray and Coxeter, that is, $d_{bc} = 3h_{bc}$, where they are using it for the axial height from one node to
the next of a different color, but we use it to mean distance for the same color.

The rail angle $\rho$ also has the meaning that $2 \pi / \rho$ is the number of
tetraheda in a full revolution of the helix.

In choosing $\rho$, one greatly constrains $r$ and $h$, but does not completely
determine both of them together, so we treat both as parameters.

Rewriting our formulation in terms of $\rho$:
\begin{equation}
H_{general}(n,c,\rho,d_{\rho},r_{\rho}) =
\left [
  \begin{tabular}{c}
   $ r_{\rho} \cos(\rho \kappa + c 2 \pi /3) $\\
   $ r_{\rho}  \sin(\rho \kappa + c 2 \pi /3) $\\
   $ d_{\rho} \kappa $
  \end{tabular}
  \right ],
\text{where:}
\begin{tabular}{c}
  $   1 = d_{\rho}^2 + 4 r_{\rho}^2 (\sin{ \rho / 2})^2 $ \\
    $\kappa = n + c/3$ \\
  \end{tabular}      
\end{equation}

$H_{general}$ generalizes $H_{continuum}$, but forces the user to select an $d_{\rho}$
which has a sensible radius, so it may be less convenient.

Note that when $\rho = 0$ then $h_{\rho} = 1$, but $r_{\rho}$ is not determined.

\begin{theorem}
  \label{generalformulaoptimal}
  The tetrahelices generated by $H_{general}$ are optimal in terms of minimum maximum member length when $r_{\rho}$ is chosen so that
  the length of the one-hop edge is equal to the rail length.
\end{theorem}

\begin{proof}
  This requires a minimax argument.
\end{proof}

By Theoerm \ref{eventhirds}, we can compute the (at most) three edge-lengths of an optimal
tetrahelix by (where $dist$ is the cartesian distance function):
\begin{align*}
  \text{rail} &= dist(H_{general}(n,c,\rho,d_{\rho},r_{\rho}),H_{general}(n+1,c,\rho,d_{\rho}),r_{\rho})) = 1 \\
  \text{one-hop} &= dist(H_{general}(n,c,\rho,d_{\rho},r_{\rho}),H_{general}(n,c+1,\rho,d_{\rho},r_{\rho}))  \\
  \text{two-hops} &= dist(H_{general}(n,c,\rho,d_{\rho},r_{\rho}),H_{general}(n,c+2,\rho,d_{\rho},r_{\rho}))  \\  
\end{align*}
Which are invarinat for all $n$ and $c$.

\begin{align*}
  \text{one-hop} &= dist(H_{general}(n,c,\rho,d_{\rho}),H_{general}(n,c+1,\rho,d_{\rho}),r_{\rho})  \\
  \text{one-hop} &= dist(H_{general}(0,0,\rho,d_{\rho}),H_{general}(0,1,\rho,d_{\rho},r_{\rho}))  \\  
  \text{one-hop}  &= \sqrt{\frac{d_{\rho}^2}{9} + r_{\rho}\sin(0) - r_{\rho}\sin(rho/3)+\frac{2\pi}{3}))^2  +
    (r_{\rho}\cos(0) - r_{\rho}\cos(\rho/3 + \frac{2\pi}{3}))^2} \\
  \text{one-hop}  &= \sqrt{\frac{d_{\rho}^2}{9} + (0  - r_{\rho}\sin(\rho/3 + \frac{2\pi}{3}))^2  +
    (r_{\rho} - r_{\rho}\cos(\rho/3 + \frac{2\pi}{3}))^2} \\
  \text{one-hop}  &= \sqrt{\frac{d_{\rho}^2}{9} + r_{\rho}^2\sin^2(\rho/3 + \frac{2\pi}{3})  +
    r_{\rho}^2(1 - \cos(\rho/3 + \frac{2\pi}{3}))^2} \\
  %% Formula above checks at sqrt(8/27), 1, 0, and rho_bc, h_bc, r_bc.
  \text{one-hop}  &= \sqrt{\frac{d_{\rho}^2}{9} + r_{\rho}^2(\sin^2(\rho/3 + \frac{2\pi}{3})  + (1 - \cos(\rho/3 + \frac{2\pi}{3}))^2)} \\
  d_{\rho}^2 &= 1 - 4 r_{\rho}^2 (\sin( \rho / 2)^2 \\
  %% Formula above checks at sqrt(8/27), 1, 0, and rho_bc, h_bc, r_bc.  
  \text{one-hop}  &= \sqrt{\frac{1}{9}  + r_{\rho}^2(-\frac{4 (\sin^2( \rho / 2))}{9} + \sin^2(\rho/3+ \frac{2\pi}{3})  + (1 - \cos(\rho/3 + \frac{2\pi}{3}))^2)} \\
  %% Formula above checks at sqrt(8/27), 1, 0, and rho_bc, h_bc, r_bc.    
\end{align*}

By similar algebra and trigonometry:
\begin{align*}
  \text{two-hops}  &= \sqrt{\frac{4}{9} + r_{\rho}^2 (-\frac{16 (\sin^2( \rho / 2))}{9} + \sin^2(2\rho/3 + \frac{4\pi}{3})  + (1 - \cos(2\rho/3 + \frac{4\pi}{3}))^2)} \\
\end{align*}

We would really like to know the partial derivative of the two-hops - one-hop with respect
to the radius to be able to understand how to choose the radius to form the minimimax optimum.

Let:
\begin{equation}
  f_{\rho} = -\frac{4 (\sin^2( \rho / 2))}{9}
  \end{equation}
\begin{equation}
  g_{\rho} = -\frac{16 (\sin^2( \rho / 2))}{9} 
\end{equation}

\begin{equation}
  j_{\rho} = \sin^2(\rho/3+ \frac{2\pi}{3})  + (1 - \cos(\rho/3 + \frac{2\pi}{3}))^2)
\end{equation}
\begin{equation}
  k_{\rho} = (\sin^2(2\rho/3 + \frac{4\pi}{3})  + (1 - \cos(2\rho/3 + \frac{4\pi}{3}))^2)
\end{equation}

Then:
\begin{align*}
  \text{two-hops} - \text{one-hop}  &= \sqrt{\frac{4}{9}  + r_{\rho}^2(g_{\rho}+ j_{\rho})}
  - \sqrt{\frac{1}{9} +r_{\rho}^2(f_{\rho}+k_{\rho}) }
\end{align*}



% Using wolfram alph we se this derivate remains negative in this range....
% plot d/dr sin(2r/3 + 4pi/3)^2 - sin(r/3 + 2pi/3)^2 + (1 - cos(2r/3 + 4pi/3))^2 -(1-cos(r/3 + 2pi/3))^2 from 0 to 35.44 degrees
% in other words

% This part is also negative withing this range...
%  d/dr -(16/9)sin(r/2)^2 - -(4/9)sin(r/2)^2 from 0 to 35.44 degrees

%% Using Mathematica, I get:
%% f[x_, r_] := 
%%  Sqrt[4/9 + 
%%     x^2 (-16 Sin[r/2]^2/9 + 
%%        Sin[2 r/3 + 4 Pi /3]^2 + (1 - Cos[2 r/3 + 4 Pi/3])^2)] - 
%%   Sqrt[1/9 + 
%%     x^2*(-4*Sin[r/2]^2/9 + 
%%        Sin[r/3 + 2 Pi/3]^2 + (1 - Cos[r/3 + 2 Pi/3])^2) ]
%% N[N[f[1, 1]]]

%% Plot3D[D[f[x, r]], {x, 0, 2}, {r, 0, Pi}]

%% This grahp demonstrates the derivative is always negative.


By graph inspection using Mathematica, we see the partial derivative of this with respect to
radius $r_{\rho}$ is always negative.
Since the partial derivative of $\text{two-hops} - \text{one-hop}$ with respect to the
radius is negative up until $\rho$ up until $\rho_{bc}$ when it is $0$, then
we will opitmize the overall minimax distance by choosing the largest radius
up until one-hop $= 1$, the rail-edge length. Numerical test suggests that this is true.


Therefore we decrease the minimax length
of the whole system as we increase the radius
up unto the point that the shorter, one-hop distance is equal to the rail-length ($1$).
Therefore, to optimize the whole system so long as $\rho \leq \rho_{bc}$,
we equate one-hop to $1$ to find the optimum radius:


\begin{align*}
  1 &=  \sqrt{\frac{1}{9}  + r_{\rho}^2(-\frac{4 (\sin^2( \rho / 2))}{9} + \sin^2(\rho/3+ \frac{2\pi}{3})  + (1 - \cos(\rho/3 + \frac{2\pi}{3}))^2)} \\
  r_{opt} &= \frac{2}{\sqrt{\frac{9}{2} \cdot ( \sqrt{3} sin(\rho/3) + \cos(\rho/3)) + cos(\rho)+ 8 }} \numberthis  \label{eqrhoopt} \\
\end{align*}
(where denominator $  \neq 0$).

Note: $ d_{opt} $ is computed by $\rho, r_{opt}$ via the rail angle equation \eqref{railangle}.

\begin{align*}
  d_{opt}^2 &= 1 - 4 \frac{2}{\sqrt{\frac{9}{2} \cdot ( \sqrt{3} sin(\rho/3) + \cos(\rho/3)) + cos(\rho)+ 8 }}^2 (\sin{ \rho / 2})^2   \\
  d_{opt}^2 &= 1 - 4 \frac{4(\sin{ \rho / 2})^2}{\frac{9}{2} \cdot ( \sqrt{3} sin(\rho/3) + \cos(\rho/3)) + cos(\rho)+ 8 }    \\
  d_{opt}^2 &= 1 - \frac{32(\sin{ \rho / 2})^2}{9 ( \sqrt{3} sin(\rho/3) + \cos(\rho/3)) + cos(\rho)+ 8 }    \\
  d_{opt}^2 &= 1 - \frac{8 sin^2(\rho / 2) }{ cos(r) + 9( \sqrt{3} sin(\rho/3) + \cos(\rho/3)) + 8 }   \\
    d_{opt} &= \sqrt{1 - \frac{8 sin^2(\rho / 2) }{ cos(r) + 9( \sqrt{3} sin(\rho/3) + \cos(\rho/3)) + 8 }}    \numberthis  \label{dopt}  \\      
\end{align*}
NOTE: I may have made a terrible mistake here, I'm unsure if the numerator should have $sin^2(\rho)$ or $sin^2(\rho/2)$.

Therefore, by computing $r_{opt}$ as a function of $\rho$ from this equation, we can construct minimax optimal tetrahelix for an $\rho < \rho_{bc}$.

\begin{theorem}
  \label{optimality}
  For any rail angle $\rho$ where $\rho \leq \rho_{bc}$, a tetrahelix of minimum maximum edge length
  difference is generated by $H_{general}(n,c,\rho,d_{\rho},r_{opt})$, where $r_{opt}$ and $d_{opt}$ are
  computed from $\rho$. 
\end{theorem}

\begin{proof}
  TBD: The Even thirds theorem tells us an optimal will have even thirds.  We have shown
  that $r_{opt}$ minimizes the maximum length.
  \end{proof}

\section{The Equitetrabeam}

Just as $H_{general}$ constructs the BC helix (with careful and non-obvious choices of parameters)
which is an important
special case due to its regularity, it constructs an additional special
(degenerate) case when the rail angle $\rho = 0$
and $d = 1$ (the edgelength), where the cross sectional area is
an equilateral triangle of unchanging orientation.
We call this the \emph{equitetrabeam}.

However, choosing $d = 1$ and $\rho = 0$ we use Equation \eqref{eqrhoopt} to find the radius of 
optimal minimax difference.

\begin{corollary}
  The equitetrabeam with minimal maximal edge difference is produced
  by $H_{general}$ when $ r = \sqrt{\frac{8}{27}} $.
  \end{corollary}

\begin{align*}
  \text{rail} &=  1 \\
  \text{one-hop} &= \sqrt{\frac{1}{9} + 3r^2}\\
\text{two-hops}    &= \sqrt{\frac{4}{9} + 3r^2}
\end{align*}
Then:
\begin{align*}
   1  &=  \sqrt{\frac{1}{9} + 3r^2} \\
   r  &= \sqrt{\frac{8}{27}} \\
   r &\approx 0.5443
\end{align*}

This radius produces a two-hop rail length of
$\frac{2}{\sqrt{3}}$. The difference between this 
and $1$ of $\approx 15.47\% $.

Note: Another interesting solution is derived by setting (one-hop + two-hop)/2 = 1,  occurs at $r = \sqrt{35}/4$,
which produces three length classes of $11/12, 12/12, 13/12$.

In Figure \ref{series}, the furthest tetrahelix is the optimal equitetrabeam.

To the extent that we value tetrabeams (that is, tetrahelicies with a rail angle of $0$,
and therefore zero curvature) as mathematical or engineering objects
we have motivated the development of $H_{general}$ as a transformation of $V(n)$ defined by
Equation \eqref{graycoxeter} from Gray and Coxeter, as it is difficult to see how
the $V(n)$ 
formulation could ever give rise to a continuum producing the tetrabeam,
since setting the angle in that equation to zero can produce only points on a single
line.

\section{An Untwisted Continuum}

We observe that Equations \eqref{eqrhoopt} and \eqref{dopt} compute $r_{opt}$ and $d_{opt}$ which
create an optimal tetrahelix for any value rail angle $\rho$ between $0$, which
gives the Equitetrahelix and
$\rho_{bc} \approx 35.43 \degree$, which gives the BC Helix.

 \begin{figure}[H]
     \centering
     \includegraphics[width=0.9\textwidth]{figures/Continuum.png}
     \caption{A Continuum of Tetrahelices}
 \end{figure}

 Negative values of $\rho$ generate clockwise tetrahelices; positive generate counter-clockwise helices.
 (Note: In fact there is something wrong with my math, negative rho values do produces opposit chirality
 but the lengths are not the same as the positive values.  I need to track this down.)

The pitch of the resulting tetrahix 
where $\rho \neq 0$ is:
\[
p(\rho) = \frac{2 \pi  \cdot d_{opt}}{\rho}
\]

As pitch increases smoothly with decreasing rail angle, it is easy to solve for rail angle
to acheive a given pitch, an ability valuable to engineers or architects who wish
to fit a tetrahelix to given dimension.

Perhaps surprisingly, the optimal untwisting is accomplished only by changing the length of
the two-hop member, leaving the one-hop memever and
rail length equivalenet within this continuum. However, it should be noted that
an engineer or architect may also use $H_{general}$ in the same way, and that minimax
length optimality is a mathematic notion only loosely related to the beauty and
utility of physical structures.

Before deriving Equation \eqref{eqrhoopt}, we created a continuum by using a linear
interpolation between the optimal radius for the Equitetrabeam and the BC Helix. This minimax
optimum of this simpler approach was at most 1\% worse than the optimum computed by
\eqref{eqrhoopt}.


\section{Utility for Robotics}

Trusses and space frames remain an important design field in
mechanical and structural engineering\cite{mikulas1985sequentially},
including deployable and moving trusses\cite{claypool2012readily}.

Starting twenty years ago, Sanderson\cite{sanderson1996modular},
Hamlin,\cite{TetrobotBook}, and others including
Lee\cite{lee2002dynamic} created a style of robotics based on changing
the lengths of members joined at the center of a joint, thereby
creating a connection to pure geometry. More recently NASA has
experimented with tensegrities\cite{NTRT}, a different point in the
same design spectrum. These fields create a need to explore the notion
of geometries changing over time, not generally considered directly by
pure geometry.

As suggested by Buckminster Fuller, the most convenient geometries to
consider are those that have regular member lengths, in order to
facilitate the inexpensive manufacture and construction of the robot.
In a plane, the octet truss is such a geometry, but in a line, the
Boerdijk--Coxeter helix is a regular structure.

However, a robot must move, and so it is interesting to consider the
transmutations of these geometries, which was in fact the motivation
for creating the equitetrabream.

\begin{theorem}
  By changing only the length of the members that change rails and make two nodes, you can untwist a tetrobot
  form the Boerdijk-Coxeter configuration to the equitetrabeam which rests flat on the plane.
\end{theorem}

\begin{proof}
  Proof by our computer program that does this by forming a linear interpolation of links.
\end{proof}

\begin{figure}[H] %float with two figures
  \centering
     \includegraphics[width=0.4\textwidth]{figures/Tetrahelix1.png}
     \caption{2/3rd Twisted Tetrahelix}
     \includegraphics[width=0.4\textwidth]{figures/Tetrahelix2.png}
     \caption{1/3rd Twisted, 2/3rd Untwisted Tetrahelix}
     \includegraphics[width=0.4\textwidth]{figures/Tetrahelix3.png}
     \caption{The Equitetrabream: Fully Untwisted Tetrahelix}
\end{figure}

\section{Conclusion}



\section{Contact and Getting Involved}

The Gluss Project \url{http://pubinv.github.io/gluss/}
is a free-libre, open-source research, hardware, and software project that welcomes volunteers.
It is our goal to organize projects for the benefit of all humanity without seeking profit or intellectual property.
To assist, contact \href{mailto:read.robert@gmail.com}{$<$read.robert@gmail.com$>$}.

\bibliographystyle{IEEEtran}
\bibliography{IEEEabrv,gluss}

\end{document}

TODO:

Clean up d_{opt}.

Produce mini table of pitch vs. \rho?

Improve clarity and precision of proofs.

Solve the chirality problem.

Check every step of my derivation at rho = 0 and the BC values.

Found something.  So now the computation of the optimal radii seems slightly better than what we had before.
We have not, however, proven it to be optimal in the ranges involved, but at least we seem to be computing it correctly.

See if we can accurately compute the derivative of two-hop - one-hop and if it is always negative, meaning that
it is always optimal to set one-hop = 1.  The we have a way to compute one-hope.

Remove the weirdly colored diagram.

Add to the rail angle diagram.

Note that Lamba = 16 almost produces a 2-rail helicoid.

When abs(lambda) > 4, we seem to have marked differences in the parameters. Understand why this is and what is causing it.
(it is possible that positive and negative rho values behave very differently?

Note: Would be really nice to have the GUI show you all computed parameters.

Note: Distance is clearly lower on first computations..

Create working toolkit for desin/exploration.

Improve code that way, use white background, floor.

Figure out if I am computing something wrong.



Try to compute optimal radius within our context. This would let us assert that we have
an ``optimum'' continuum.  Not worth much to an engineer, but valuable.

Review and continue editing, being careful to introduce concepts in the correct order.

On Rail diagram, add a second tetrahedron, and a second rho.



Make YouTube video.


    \section{Old Proofs}
    
We can create the distance between representative nodes as:
\begin{align*}
  \overline{\rm AB}  &= 1 \\        
  \overline{\rm CD}  &= \sqrt{G^2 + r^2(1 + -\cos{2 G \rho})} \\
  \overline{\rm AC}  &= \sqrt{O^2 + r^2(1 + \sin^2{O \rho}  - \cos^2{O \rho})} \\  
  \overline{\rm AC}  &= \sqrt{O^2 + r^2(1 + -\cos{2 O  \rho})} \\
  \overline{\rm BD}  &= \sqrt{P^2 + r^2(1 + -\cos{2 P \rho}))} \\
  \overline{\rm AD}  &= \sqrt{(O+G)^2 + r^2(1 + -\cos{2 (O + G)  \rho})} \\
  \overline{\rm BC}  &= \sqrt{(P+G)^2 + r^2(1 + -\cos{2 (P + G) \rho})} \\      
\end{align*}


Using the cartesian distance formula:
\begin{align*}
  \overline{\rm CD}  &= \sqrt{G^2 + (r\sin{0 \rho} -  r\sin{G \rho})^2 + (r\cos{0 \rho} - r\cos{G \rho})^2} \\
    \overline{\rm CD}  &= \sqrt{G^2 + (0 -  r\sin{G \rho})^2 + (r - r\cos{G \rho})^2} \\
    \overline{\rm CD}  &= \sqrt{G^2 + r^2(\sin{G \rho})^2 + r^2(1 - \cos{G \rho})^2} \\
    \overline{\rm CD}  &= \sqrt{G^2 + r^2(1 + \sin^2{G \rho}  - \cos^2{G \rho})} \\
    \overline{\rm CD}  &= \sqrt{G^2 + r^2(1 + -\cos{2 G \rho})} \\        
\end{align*}
Note: $\sin^2{x \rho}  - \cos^2{x \rho} = -\cos{2 x \rho}$

So, for all O,G,P,
\begin{align*}
  \overline{\rm AB}  &= 1 \\        
  \overline{\rm CD}  &= \sqrt{G^2 + r^2(1 + -\cos{2 G \rho})} \\
  \overline{\rm AC}  &= \sqrt{O^2 + r^2(1 + \sin^2{O \rho}  - \cos^2{O \rho})} \\  
  \overline{\rm AC}  &= \sqrt{O^2 + r^2(1 + -\cos{2 O  \rho})} \\
  \overline{\rm BD}  &= \sqrt{P^2 + r^2(1 + -\cos{2 P \rho}))} \\
  \overline{\rm AD}  &= \sqrt{(O+G)^2 + r^2(1 + -\cos{2 (O + G)  \rho})} \\
  \overline{\rm BC}  &= \sqrt{(P+G)^2 + r^2(1 + -\cos{2 (P + G) \rho})} \\      
\end{align*}




Then we can simplify our distance equations:
\begin{align*}
  \overline{\rm AB}  &= 1 \\        
  \overline{\rm CD}  &= \sqrt{G^2 + r^2(1 + -\cos{2 G \rho})} \\
  \overline{\rm AC} &=   \overline{\rm BD} &= \sqrt{P^2 + r^2(1 + -\cos{2 P \rho}))} \\
  \overline{\rm AC} &=   \overline{\rm BD} &= \sqrt{((1-G)/2)^2 + r^2(1 + -\cos{ \rho (1-G) }))} \\  
  \overline{\rm AD} &=  \overline{\rm BC} &=  \sqrt{((P+G)^2 + r^2(1 + -\cos{2 (P + G) \rho})} \\
  \overline{\rm AD} &=  \overline{\rm BC} &=  \sqrt{((1+ G)/2 )^2 + r^2(1 + -\cos{\rho (1 + G)})} \\        
\end{align*}

When $G = 0$, $\overline{\rm CD} = 0$. Our solution is improved as we increase $G$ until either $ \overline{\rm CD} = \overline{\rm AC}$, or the
quantity $ \overline{\rm BC} - \overline{\rm CD} $ starts increasing rather than decreasing (that is, when the derivative is positive.)



Suppose $\overline{\rm C'D'}$ is the shortest edge. Increasing $G$ thereby improves our mimimum, up until the next shortest edge length, $\overline{\rm AC}$.
This may increase $\overline{\rm BC}$ and $\overline{\rm AD}$, but more slowly (???) than $\overline{\rm C'D'}$ is being increased, until  $\overline{\rm C'D'} =
\overline{\rm AC}$.
Note: This is sketchy, can we show that the derivative of our formula for CD  with respect to $G$ is really higher than $AD$? These derivatives will depend on $\rho$ slightly, which is exactly right!

\begin{align*}
  \text{rail} &=  1 \\
  \text{one-hop} &= \sqrt{\frac{1}{3}^2 + (\frac{3a}{2\sqrt{3}})^2 + (a/2)^2}\\
\text{one-hop}  &= \sqrt{\frac{1}{9} + a^2} \\
    \text{two-hops} &= \sqrt{\frac{2}{3}^2 + (\frac{3a}{2\sqrt{3}})^2 + (a/2)^2}  \\
\text{two-hops}    &= \sqrt{\frac{4}{9} + a^2}
\end{align*}
Computing the derivative of two-hops - one-hop:
\begin{align*}
 \frac{\partial \text{two-hops} - \text{one-hop}}{a} &= \frac{\partial \sqrt{4/9 + a^2} - \sqrt{1/9 + a^2}}{\partial a} \\
  \frac{\partial \text{two-hops} - \text{one-hop}}{a} &= \frac{a}{\sqrt{a^2 + 4/9}} - \frac{a}{\sqrt{ a^2 + 1/9}} 
\end{align*}

  
Now basic components of the helix, which are the radius $r$, the rate of rotation, and the rate of
axial growth can all be linearly interpolated with a parameter $\lambda$ between their high values (for the BC helix)
and low values (for the equitetrabeam):

\begin{align*}
r_{\lambda}  &=  \lvert \lambda \rvert \cdot (\frac{3 \sqrt{3}}{10}  - \sqrt{\frac{8}{27}}) + \sqrt{\frac{8}{27}}  \\
d_{\lambda} &=   \lvert \lambda \rvert \cdot (3 \sqrt{10} - 1) + 1 \\
\phi_{\lambda} &=  \lambda \cdot \rho_{bc}  + 0
\end{align*}
to create a formula that generates a continuum of tetrahedral structures:

\[
H_{continuum}(n,c,\lambda) =
\left [
  \begin{tabular}{c}
   $ r_{\lambda} \cos(\phi_{\lambda} \kappa + c 2 \pi /3) $\\
   $ r_{\lambda}  \sin(\phi_{\lambda} \kappa + c 2 \pi /3) $\\
   $ d_{\lambda} \kappa $
  \end{tabular}
  \right ],
\text{where:}
  \begin{tabular}{c}
    $\kappa = n + c/3$ \\
    $ r_{opt} =  \text{is used : \eqref{eqrhoopt}} $ \\    
    $ \phi_{\lambda} =  \lambda \rho_{bc}  + 0 $\\
    $\rho_{bc} = (3 \theta - 2 \pi)$ \\
   $ \theta = \arccos(-2/3) $
  \end{tabular}      
\]

  
